\section{Encoding: 10'}



\begin{frame}[fragile]{Encoding: Goal}
\begin{itemize}
\item A string is not a sequence of bytes
\item A string is not a sequence of bytes
\item Manage differently encoded filenames 
\item A string is not a sequence of bytes
\item modules: \emph{os, os.path, glob}
\end{itemize}
\end{frame}


\begin{frame}{Song of Childhod}
\begin{verse}
\begin{center}
\emph{```When the child was a child,\\
\\
\\
characters were bytes, and\\
\\
\\
strings list of bytes'''}

\end{center}
\end{verse}
\end{frame}

\begin{frame}[fragile]{Enters Encoding}
\begin{itemize}

\item Encoding is a one-to-one map between a typographical character and a byte-sequence - Decoding is the reverse map.
\begin{tabular}{|c||c|c|c|}\hline 

            & \multicolumn{3}{|c|}{bytes}  \\ \hline
char        & ascii     & utf-8         & cp1252     \\ \hline
a           & [97]      & [97]          & [97]      \\ \hline     
$\ddot{u}$  & -         & [195, 188]    & [252]              \\ \hline
\`{e}     &  - & [196, 168] & [232]\\ \hline
\end{tabular}
\item The \emph{type()} of a byte-sequence is bytes

\item A string is a couple: (bytes, encoding) - the same string can be encoded using different maps.
\begin{tabular}{|c|l|} \hline 
encoding & the string  w\emphred{\"u}rstel results in bytes \\ \hline 
utf-8 &([119, \emphred{195, 188,} 114, 115, 116, 101, 108]  \\
cp1252 &([119, \emphred{252,} 114, 115, 116, 101, 108]\\
\hline
\end{tabular}

\end{itemize}
\end{frame}


\begin{frame}[fragile]{Enters Encoding}
\begin{verse} \begin{center}
\Huge
`\`e'  or `\~{A}\"{ }'
\\
\end{center} \end{verse}

\begin{center}
\huge
\`{e} $\xmapsto[encode]{utf-8}$ 
    [\red{196}, \blue{168}] 
    $\xmapsto[decode]{cp1252}$ 
    \red{\~{A}} \blue{\"{ }}
\end{center}

\end{frame}

\begin{frame}[fragile]{Enters Encoding II}
\begin{minted}[mathescape]{python}
# Filenames are binary data! $\emph{Be careful}$ when reading from
#  a (eg. vfat) filesystem!
# To make python2 encoding-aware we should
from __future__ import unicode_literals

# Create 3 windows-encoded filenames in 
basedir = "/tmp/course"

# using the provided function
from exercises import create_wuerstelstrasse
create_wuerstelstrasse(basedir)
\end{minted}
\end{frame}



\begin{frame}[fragile]{Encoded filenames}
\begin{minted}[mathescape]{python}
# What happens mangling them with the $\emph{os}$ module?
from os.path import join as pjoin
from os import listdir as ls       # similar to unix ls ;)

list_files_with = ls(basedir)

# ooops! Got an encoding error?
create_full_path = pjoin(basedir, files[0])

# $\emphred{UnicodeDecodeError:}$ 'ascii' codec can't decode $\pyver{byte 0xfc}$
#    in position 2: ordinal not in range(128)
0xfc == 252 # remember the $\"u$ in cp1252 map? 
\end{minted}
\end{frame}

\begin{frame}[fragile]{Encoded filenames II}
\begin{minted}[mathescape]{python}

# $\emph{os.path}$ fails mangling those files as encoded strings
#  so we ask it to mangle them as $\emph{bytes}$
bytebasedir = bytes(basedir)

for f in files: # listdir is already safe ;)
    byte_full_path = pjoin(basedir, f)
    # we use "{!r}".format to avoid further encoding
    #     issues in the printing part
    print("file: {!r}".format(utf_path))
    
\end{minted}
\end{frame}

\begin{frame}[fragile]{Encoded filenames: glob}
\begin{minted}[mathescape,fontsize=\footnotesize]{python}
# $\emph{glob.glob}$ expands wildcards like a shell. 
from glob import glob

# To avoid encoding issues like the following...
try:
    files = glob("/tmp/course/*.txt")
except UnicodeDecodeError as e:
    print("Error decoding files in {!r}".format("/tmp/course"))

# ...we explicitly ask to use byte-sequences
for f in glob(b"/tmp/course/*.txt"):
    try:
        print("file: {!r}".format(f))
    except UnicodeDecodeError as e:
        print("Error decoding {!r}".format(f))

\end{minted}
\end{frame}


\begin{frame}[fragile]{Encoded filenames: Complete Example}
\begin{minted}[mathescape]{python}
def list_files(basedir):
    """Works both if isinstance(basedir, unicode)
        or isinstance(basedir, bytes)
    """"
    for f in ls(basedir):
        try:
            utf_or_byte_path = pjoin(basedir, f)
            print("file: {!r}".format(utf_or_byte_path))
        except UnicodeDecodeError as e:
            print("Error decoding {!r}".format(f))

bytebasedir = bytes(basedir)
list_files(basedir)     # which one ...
list_files(bytebasedir) # ...will work?
    
\end{minted}
\end{frame}
