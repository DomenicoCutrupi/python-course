\section{ipython}



\begin{frame}{iPython I}
\begin{itemize}
\item Interactive interpreter with tonns of                                                                                                        functionalities, and the main tool of our training.                                                                                                      

\item   The funniest way to learn and use python!

\item   Supports \pyver{tab-completion}, inline help 

\item   Allows pasting from clipboard with \pyver{\%paste} , elementary
    benchmarking with \pyver{\%timeit}, and multi-line editing with \pyver{\%edit}

\item   Run it: \\
\code{\# ipython}
\end{itemize}

\end{frame}


\begin{frame}[fragile]{iPython II}
\begin{pythoncode}
# iPython supports inline-help appending ? to an object
str?

# We can run commands and capture the output in a variable
# don't need to quote using the ! magic on unix
ret = !cat /etc/hosts

# windows has etc\hosts too ;)
ret = !type c: windows\system32\drivers\etc\hosts
\end{pythoncode}
\end{frame}


\begin{frame}[fragile]{iPython III}
\begin{pythoncode*}{escapeinside=||}
# returned objects can be filtered with  
ret.|\pyver{grep}|('localhost')
# Now get the first space-splitted column of the output
ret.|\pyver{fields}|(0)
ret.grep('localhost').fields(0)

# And the last returned value is stored in 
localip = _

# An ipython script $\emph{must}$ have the $\pyver{.ipy}$ extension
#  and begin with #!/usr/bin/ipython 
\end{pythoncode*}
\end{frame}

