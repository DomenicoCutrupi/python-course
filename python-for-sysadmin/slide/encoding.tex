\section{Encoding: 10'}

\begin{pyframe}{Encoding: Goal}
\Large
\begin{itemize}
\item A string more than a sequence of bytes
\item \emph{A string is a couple (bytes, encoding)}
\item Use unicode\_literals in python2
\item Manage differently encoded filenames
\item A string is not a sequence of bytes
\end{itemize}
modules: \pymodule{os, os.path, glob}
\end{pyframe}


\begin{pyframe}{Song of \emph{C}hildhood}
\begin{columns}

\column[c]{.15\textwidth}
\tiny
\textit{Als das Kind Kind war, ging es mit hängenden Armen,
wollte der Bach sei ein Flu\ss, der Flu\ss sei ein Strom,
und diese Pf\"utze das Meer.
\newline
Als das Kind Kind war,  wu\sste es nicht, da\ss es Kind war,
alles war ihm beseelt, und alle Seelen waren eins.
\newline
Als das Kind Kind war,
hatte es von nichts eine Meinung,
hatte keine Gewohnheit,
sa\ss oft im Schneidersitz,
lief aus dem Stand,
hatte einen Wirbel im Haar
und machte kein Gesicht beim fotografieren.
}
\column[c]{.7\textwidth}
\begin{verse}
\begin{center}
\Large
```When the child was a child,\\
\vspace{.5cm}
characters were bytes, and\\
\vspace{.5cm}
strings list of bytes'''
\end{center}
\end{verse}

\column[c]{.15\textwidth}
\tiny \textit{Als das Kind Kind war,
fielen ihm die Beeren wie nur Beeren in die Hand
und jetzt immer noch,
machten ihm die frischen Waln\"usse eine rauhe Zunge
und jetzt immer noch,
hatte es auf jedem Berg
die Sehnsucht nach dem immer h\"oheren Berg,
und in jeder Stadt
die Sehnsucht nach der noch gr\"o\sseren Stadt,
und das ist immer noch so,
griff im Wipfel eines Baums nach dem Kirschen in einemHochgef\"uhl
wie auch heute noch,
eine Scheu vor jedem Fremden
und hat sie immer noch,
wartete es auf den ersten Schnee,
und wartet so immer noch.
}
\end{columns}
\end{pyframe}

\iffalse
\begin{pyframe}{Encoding is a map}
\begin{itemize}
\item The \emph{type()} of a byte-sequence is bytes
\item Encoding is a one-to-one map between a typographical character and a byte-sequence
\item Decoding is its reverse map
\end{itemize}

\begin{columns}
\column[c]{.6\textwidth}
\begin{pycode*}{escapeinside=||}
# heading $\pyver{u}$ only py2 and py3.3+
b1 = |\pyver{u}|"S\u00fcd".encode('utf-8')
b2 = |\pyver{u}|"S\u00fcd".encode('cp1252')

# it's always true that
type(b1) == type(b2) == bytes
# but only in py2
str == bytes





\end{pycode*}

\column[c]{.4\textwidth}
\small
\begin{tabular}{|c||c|c|c|}\hline
            & \multicolumn{3}{|c|}{bytes}  \\ \hline
char        & ascii     & utf-8         & cp1252     \\ \hline
a           & [97]      & [97]          & [97]      \\ \hline
$\ddot{u}$  & -         & [195, 188]    & [\pyver{252}]              \\ \hline
\`{e}     &  - & [196, 168] & [232]\\ \hline
\end{tabular}
\end{columns}

\end{pyframe}
\fi
%%
\begin{pyframe}{Encoding is a map}
\begin{columns}
\column[c]{.6\textwidth}


\begin{pycode*}{escapeinside=||}
# Py3 doesn't need the $\pyver{u}$
the_string = |\pyver{u}|"S\u00fcd" # $S\ddot{u}d$

# can be encoded in different
in_utf8 =  the_string.encode('utf-8')
in_win = the_string.encode('cp1252')

type(in_utf8) == bytes # byte-sequences

# Decoding bytes using the wrong map..
# ...gives $sad$ results ;)
in_utf8.decode('cp1252') # $S\tilde{A}\sfrac{1}{4}d$





\end{pycode*}

\column[c]{.4\textwidth}
\begin{itemize}
\item Encoding is a one-to-one map between a typographical character and a byte-sequence
\item Decoding is its reverse map
%\item The \emph{type()} of a byte-sequence is bytes
\end{itemize}

\small
\begin{tabular}{|c||c|c|c|}\hline
%            & \multicolumn{3}{|c|}{bytes}  \\ \hline
char        & ascii     & utf-8         & cp1252     \\ \hline
a           & [97]      & [97]          & [97]      \\ \hline
$\ddot{u}$  & -         & [195, 188]    & [252]              \\ \hline
%\`{e}     &  - & [196, 168] & [232]\\ \hline
\end{tabular}
\end{columns}

\end{pyframe}



%%

\iffalse
\begin{pyframe}{De}
\begin{itemize}
\item A string is a couple: (bytes, encoding)
\item The same string can be encoded using different maps.
\end{itemize}

\begin{table}
\begin{tabular}{|c|l|} \hline
encoding & the string  S\pyver{\"u}d results in bytes \\ \hline
utf-8 &([83, \pyver{195, 188}, 100]  \\
cp1252 &([83, \pyver{252}, 100]\\
\hline
\end{tabular}
\end{table}

\begin{verse} \begin{center}
\huge
\"u  {\footnotesize versus}  \~{A}  $\sfrac{1}{4}$
\\
\end{center} \end{verse}

\begin{center}
\Large
\"{u} $\xmapsto[encode]{utf-8}$
    [\red{198}, \blue{188}]
    $\xmapsto[decode]{cp1252}$
    \red{\~{A}} \blue{$\sfrac{1}{4}$}
\end{center}

\end{pyframe}
\fi

\begin{pyframe}{Enters Encoding}
\begin{minted}[mathescape]{python}
# Filenames are binary data! $\emph{Be careful}$ when reading from
#  a (eg. vfat) filesystem!
# To make python2 encoding-aware we should
from __future__ import unicode_literals

# Create 3 windows-encoded filenames in
basedir = "/tmp/py"

# using the provided function
from course import create_wuerstelstrasse
create_wuerstelstrasse(basedir)
\end{minted}
\end{pyframe}


\iffalse
\begin{pyframe}{Encoded filenames}
\begin{pycode*}{escapeinside=||}
# What happens mangling them with the $\pymodule{os}$ module?
from os.path import join as pjoin
from os import listdir as ls       # similar to unix ls ;)

list_files_with = ls(basedir)

# ooops! Got an encoding error?
create_full_path = pjoin(basedir, list_files_with[0])

# $\emphred{UnicodeDecodeError:}$ 'ascii' codec can't decode $\pyver{byte 0xFC}$
#    in position 2: ordinal not in range(128)
|\pyver{0xFC}| == 252 # remember the $\ddot{u}$ in cp1252 map?
\end{pycode*}
\end{pyframe}

\begin{pyframe}{Encoded filenames II}
\begin{pycode*}{escapeinside=||}

# $\pymodule{os.path}$ on py2 fails mangling those files as encoded strings
#  so we ask it to mangle them as $\emph{bytes}$
bytebasedir = bytes(basedir)

for f in files: # listdir is already safe ;)
    byte_full_path = pjoin(basedir, f)
    # we use $\pyver{"{!r}"}$.format to avoid further encoding
    #     issues when printing or logging
    print("file: {!r}".format(utf_path))

\end{pycode*}
\end{pyframe}
\fi

\begin{pyframe}{Encoded filenames: glob}
\begin{pycode*}{escapeinside=||}
from glob import glob as ls # expands wildcards like a shell.

files = ls("/tmp/py/*.txt") # To avoid encoding issues ...
# $\red{UnicodeDecodeError:}$ 'ascii' codec can't decode $\pyver{byte 0xFC}$
|\pyver{0xFC}| == 252 # remember the $\ddot{u}$ in cp1252 map?

files = ls(|\pyver{b}|"/tmp/py/*.txt") #..we explicitly use bytes

\end{pycode*}
\end{pyframe}

\iffalse
\begin{pyframe}{Encoded filenames: Complete Example}
\begin{pycode}
def list_files(basedir):
    """Works both if isinstance(basedir, unicode)
        or isinstance(basedir, bytes)"""
    for f in ls(basedir):
        try:
            utf_or_byte_path = pjoin(basedir, f)
            print("file: {!s}".format(utf_or_byte_path))
        except UnicodeDecodeError as e:
            print("Error decoding {!r}".format(f))

bytebasedir = bytes(basedir)
list_files(basedir)     # which one ...
list_files(bytebasedir) # ...will work?

\end{pycode}
\end{pyframe}
\fi
